\documentclass[12pt]{beamer}

%PTBR
\usepackage[brazilian]{babel}
\usepackage[utf8]{inputenc}
\usepackage[T1]{fontenc}
\usepackage{appendixnumberbeamer} %frames sem número

    
\usepackage{amssymb, wasysym}
\usepackage{amsmath}
\usepackage{turnstile}% http://ctan.org/pkg/turnstile
\usepackage{tikz}% http://ctan.org/pkg/pgf
\usepackage{adjustbox}% http://ctan.org/pkg/adjustbox
\usepackage{proof}
\usepackage{pgfgantt}

\newcommand\sqsubsetap{\mathrel{\substack{
  \textstyle\sqsubset\\[-0.2ex]\textstyle\sim}}}
\renewcommand{\makehor}[4]
  {\ifthenelse{\equal{#1}{n}}{\hspace{#3}}{}
   \ifthenelse{\equal{#1}{s}}{\rule[-0.5#2]{#3}{#2}}{}
   \ifthenelse{\equal{#1}{d}}{\setlength{\lengthvar}{#2}
     \addtolength{\lengthvar}{0.5#4}
     \rule[-\lengthvar]{#3}{#2}
     \hspace{-#3}
     \rule[0.5#4]{#3}{#2}}{}
   \ifthenelse{\equal{#1}{t}}{\setlength{\lengthvar}{1.5#2}
     \addtolength{\lengthvar}{#4}
     \rule[-\lengthvar]{#3}{#2}
     \hspace{-#3}
     \rule[-0.5#2]{#3}{#2}
     \hspace{-#3}
     \setlength{\lengthvar}{0.5#2}
     \addtolength{\lengthvar}{#4}
     \rule[\lengthvar]{#3}{#2}}{}
   \ifthenelse{\equal{#1}{w}}{% New wavy $\sim$ definition
     \setbox0=\hbox{$\sim$}%
     \raisebox{-.6ex}{\hspace*{-.05ex}\adjustbox{width=#3,height=\height}{\clipbox{0.75 0 0 0}{\usebox0}}}}{}
   \ifthenelse{\equal{#1}{z}}{% New tikz wavy definition
     \raisebox{-.4ex}{\hspace*{-.08ex}\tikz \draw [thin,x=0.5ex,y=.25ex] (0,0) sin (1,1) cos (2,0) sin (3,-1) cos (4,0);\hspace*{.2ex}}}{}   
  }


%Meus comandos para letras de KB
\newcommand{\T}{\mathcal{T}}
\newcommand{\D}{\mathcal{D}}
\newcommand{\B}{\mathfrak{B}}
\newcommand{\dvd}{\turnstile{s}{w}{}{}{}}
\newcommand{\redkb}{\langle\overset{\sim}{\T},\overset{\sim}{\Delta}\rangle}
\newcommand{\dkb}{\langle\T,\B\rangle}

\newcommand{\definc}{\mathrel{\substack{
\textstyle\sqsubset\\[-0.2ex]\textstyle\sim}}}

\newcommand*{\defvdash}{~\mid\!\sim} %defvdash

%newcommand*{\defvdash}{\ensuremath{\mathrel{\medvert\mskip-5.7mu\clipbox{1 0 0 0}{$~\mid\!\sim$}}}} %defvdash

\usetheme[progressbar=frametitle]{metropolis}
\usepackage{appendixnumberbeamer}

\usepackage{booktabs}
\usepackage[scale=2]{ccicons}

\usepackage{pgfplots}
\usepgfplotslibrary{dateplot}

\usepackage{xspace}
\newcommand{\themename}{\textbf{\textsc{metropolis}}\xspace}

\title{Topics in Description Logics of Typicality and Quantification Neglect}
\date{15 de Dezembro de 2020}
%\date{26 de Março de 2019}
\author{Candidato: Igor de Camargo e Souza Câmara \\
Orientadora: Profa. Dra. Renata Wassermann}
\institute{IME - USP}
% \titlegraphic{\hfill\includegraphics[height=1.5cm]{logo.pdf}}

\begin{document}

%\maketitle

\appendix %sem numeração

\begin{frame}
\frametitle{Typicality and Description Logics}
\begin{columns}
\begin{column}{0.6\textwidth}
   ~\\
   { 
   \footnotesize \textbf{Typicality}: phenomenon of human cognition. %well-studied phenomenon of human cognition.
\begin{itemize}
    \itemsep-0.3em 
    \item “Birds (typically) fly.”
    \item Relates to non-monotonicity.
    \item \textit{x} is a bird implies \textit{x} flies. \textit{x} is also a penguin, therefore \textit{x} does not fly.
\end{itemize}

\textbf{A Survey of DLs of Typicality}

%230 papers, 231 authors.
%\textbf{Goals}: 
\begin{itemize}
    \itemsep-0.3em 
    \item Identify the main tendencies and pinpoint open-problems.
    \item Representing typicality (language level).
    \item Adapting DL’s semantics.
    \item Reasoning techniques dealing with typical information.
\end{itemize}
}
\end{column}

\begin{column}{0.55\textwidth}  %%<--- here
%\begin{center}
    
{\footnotesize 
    
\textbf{3 main “schools”}
\vspace{-0.7em}
\begin{itemize}
    \itemsep-0.3em
    \item DLs with modal-like typicality operators. %\textit{\textbf{T}(Penguin)} denotes typical penguins.
    \item Defeasible DLs. %$Bird \definc Fly$ means that birds usually fly.
    \item DL + Rules.
\end{itemize}

\vspace{-0.7em}
\textbf{Semantics} equipped with preference relations.

\textbf{Reasoning techniques}
\vspace{-0.7em}
\begin{itemize}
    \itemsep-0.3em
    \item Tableaux. %that record preferential relations.
    \item Materialisation and closures.
\end{itemize}

\textbf{Some problems}:
Inheritance blocking.
Context and single preference orderings.
\underline{Quantification neglect}.
}
    
%\end{center}
\end{column}
\end{columns}
\end{frame}



\begin{frame}
\frametitle{Quantification Neglect and Typicality Models}
\begin{columns}
\begin{column}{0.6\textwidth}

{\footnotesize 
\textbf{Quantification neglect}
\vspace{-0.7em}
\begin{itemize}
    \itemsep-0.3em
    \item Reasoning procedures of propositional nature.
    \item $Bird(x)$ implies $Fly(x)$
    \item $\exists eats.Bird$ does not imply $\exists eats.Fly$.
\end{itemize}

\textbf{Typicality Models}
\vspace{-0.7em}
\begin{itemize}
    \itemsep-0.3em
    \item Canonical model construction for defeasible $\mathcal{EL}_\bot$.
    \item Domain elements: $(C, \mathcal{U})$, where $C$ is a concept and $\mathcal{U} \sqsubseteq \mathcal{D}$.
    %\item Reasoning procedures (\textit{e.g.} rational), individuals introduced by existential quantification are of the form: $(C, \emptyset)$.
    %\item Procedures to upgrade these individuals.
    \item Individuals introduced by existential quantification are atypical. Typicality models provide tools to upgrade them. %amend that by upgrading the defeasible information conveyed by them.
\end{itemize}

}
\end{column}

\begin{column}{0.55\textwidth}  %%<--- here
    %\begin{center}
    
{\footnotesize 
    \textbf{My goals}
    \vspace{-0.7em}
    \begin{itemize}
        \item Port typicality models to more expressive defeasible DLs.
        \item Investigate it for DLs with the \textit{canonical model property}. 
    \end{itemize}
    
    \textbf{Current challenges}
    \vspace{-0.7em}
    \begin{itemize}
        \item Construction relies on operations not necessarily available to all canonical models.
        \item \textbf{Subgoal}: find suitable canonical model constructions to classes of more expressive DLs.
    \end{itemize}
}
    
%\end{center}
\end{column}
\end{columns}
\end{frame}

\end{document}




